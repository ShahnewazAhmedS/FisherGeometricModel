\documentclass[a4paper]{article}
%\usepackage{simplemargins}

%\usepackage[square]{natbib}
\usepackage{amsmath}
\usepackage{amsfonts}
\usepackage{amssymb}
\usepackage{graphicx}

\begin{document}
\pagenumbering{gobble}

\Large
 \begin{center}
 Understanding the Distribution of Fitness Effects (DFE) Across Environments: exploring the Statistical Properties of Fitness Landscapes Using the Fisher Geometric Model\\
\hspace{10pt}

% Author names and affiliations
\large
Shahnewaz Ahmed$^1$, Michael Manhart$^2$ \\

\hspace{10pt}

\small  
$^1$ Department of Physics and Astronomy\\
$^2$ Center for Biotechnology and Advanced Medicine \\
Rutgers University

\end{center}

\hspace{10pt}

\normalsize

The distribution of fitness effects (DFE) is a fundamental concept in evolutionary biology; understanding the DFE is important for analyzing a variety of phenomena, including quantitative traits, complex diseases, and the evolution of antibiotic resistance. The DFE describes how mutations affect an organism's fitness in a specific environment. However, since environments in nature are constantly changing, understanding how the DFE varies across different environments is crucial for predicting evolutionary outcomes. In this study, we employ the Fisher Geometric Model (FGM) to explore the statistical properties of DFEs and their dependence on environmental changes. The FGM provides a theoretical framework to model fitness landscapes, where mutations are represented as random perturbations in a high-dimensional phenotypic space, and fitness is determined by the distance to an optimal phenotype.

We compare the predictions of the FGM with empirical data from a large-scale study of ~3800 gene knockout mutants of \textit{E. coli} across ~100 environments. These environments include both stress conditions (e.g., antibiotics, metals, salts) and non-stress conditions (varying carbon and nitrogen sources). By simulating DFEs under different environmental conditions, we investigate whether the FGM can reproduce key statistical features observed in the data, such as the bimodal distribution of fitness means and variances. Our results suggest that while the FGM captures some aspects of the data, it struggles to fully explain the observed patterns, particularly the correlations between environments. This work highlights the limitations of the FGM in describing empirical data and underscores the need for alternative models.


\end{document}